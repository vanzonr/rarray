\documentclass[11pt,twoside]{article}
\usepackage{a4wide,fancyhdr,hyperref,framed,charter}
\setcounter{tocdepth}{1}
\pagestyle{fancy}
\renewcommand{\sectionmark}[1]{\markright{\thesection\quad #1}}
\fancyhf{}
\fancyhead[LE,RO]{\bfseries\thepage}
\fancyhead[LO]{\bfseries \hfill\rightmark\hfill}
\fancyhead[RE]{\hfill\bfseries\rightmark \hfill}
\renewcommand{\headrulewidth}{0.5pt}
\renewcommand{\footrulewidth}{0pt}
\addtolength{\headheight}{2.5pt}
\fancypagestyle{plain}{\fancyhead{}\renewcommand{\headrulewidth}{0pt}}
\let\oldtt=\tt
%\renewcommand{\texttt}[1]{\color{blue}\tt #1\color{black}}
\newcommand{\cxx}{C{++}}

%\renewcommand{\texttt}[1]{#1}
\begin{document}

\setlength{\parskip}{1mm}

\title{\texttt{rarray}: Multidimensional Runtime Arrays for \cxx}

\author{Ramses van Zon%\\
%\it\small SciNet High Performance Computing Consortium, University
%of Toronto, Toronto, Ontario, Canada
\vspace{-8pt}} 

\date{January 4, 2015\vspace{-7mm}}

\maketitle

\section{For the impatient: the what, why and how of rarray}

\noindent\textbf{What:}

Rarray provides multidimensional arrays with dimensions determined at runtime. 

\noindent\textbf{What not:} 

No linear algebra, overloaded operators etc.

\noindent\textbf{Why:} 

Usually faster than alternatives,

Uses the same accessors as compile-time (automatic) arrays,

Data is guarranteed to be contiguous for easy interfacing with libraries.

\noindent\textbf{How:}

The header file \texttt{rarray.h} provides the type \texttt{rarray<T,R>}, where \texttt{T} is any type and {\tt R} is the rank. Element access uses repeated square brackets. Copying rarrays or passing them to functions mean shallow copies, unless explicitly asking for a deep copy.

\

\centerline{\begin{tabular}{|l|l|}
\hline
\rule{0pt}{14pt}Define a \texttt{n$\times$m$\times$k} array of \texttt{float}s:&
\texttt{rarray<float,3> b(n,m,k);}
\\
\rule{0pt}{14pt}Define a \texttt{n$\times$m$\times$k} array of \texttt{float}s with&
\\
data already allocated at a pointer \texttt{ptr}:&
\texttt{rarray<float,3> c(ptr,n,m,k);}
\\
\rule{0pt}{14pt}Element i,j,k of the array \texttt{b}:&
\texttt{b[i][j][k]}
\\
\rule{0pt}{14pt}Pointer to the contiguous data in \texttt{b}:&
\texttt{b.data()}
\\
\rule{0pt}{14pt}Extent in the \texttt{i}th dimension in \texttt{b:}&
\texttt{b.extent(i)}
\\
\rule{0pt}{14pt}Shallow copy of the array:&
\texttt{rarray<float,3> d=b;}
\\
\rule{0pt}{14pt}Deep copy of the array:&
\texttt{rarray<float,3> e=b.copy();}
\\
\rule{0pt}{14pt}A rarray using an existing automatic array:&
\texttt{float f[10][20][8]=\{\dots\};}\\
&
\texttt{rarray<float,3> g=RARRAY(f);}
\\
\rule{0pt}{14pt}A rarray copy of an existing automatic array:&
\texttt{rarray<float,3> h=RARRAY(f).copy();}
\\
\rule{0pt}{14pt}Output a rarray to screen:&
\texttt{std::cout << h << endl;}
\\
\rule{0pt}{14pt}Read a rarray from keyboard:&
\texttt{std::cin >> h;}
\\\hline
\end{tabular}}



\section{Introduction}

While C and thus C++ has some support for multidimensional arrays whose sizes are known at compile time, the support for arrays with sizes that are known only at runtime, is limited. For one-dimensional  arrays,  C++ has a reasonable allocation construction in the operators \texttt{new} and \texttt{delete}. An example of the standard way to allocate a one-dimensional array is the following piece of code:
\vspace{-5pt}\begin{framed}\vspace{-14pt}%
\begin{verbatim}
float* a;
int n = 1000;
a = new float[n];
a[40] = 2.4;
delete[] a;
\end{verbatim}%
\vspace{-12pt}\end{framed}\vspace{-5pt}%
It is important to note that this code also works if \texttt{n} was not known yet, e.g., if it was passed as a function argument or read in as input. 

In the above code snippet, the new/delete construct assigns the address of the array to a pointer. This pointer does not remember its size, so this is not really an 'array'.  The standard C++ library does provide a one-dimensional array that remembers it size in the form of the \texttt{std::vector}, e.g.
\vspace{-5pt}\begin{framed}\vspace{-14pt}%
\begin{verbatim}
std::vector a;
int n = 1000;
a.reserve(n);
a[40] = 2.4;
a.clear();
\end{verbatim}%
\vspace{-12pt}\end{framed}\vspace{-5pt}%

Multi-dimensional runtime-allocated arrays are not supported by \cxx.
The textbook \cxx\ solution for multidimensional arrays that are
dynamically allocated during runtime, is as follows:
\vspace{-5pt}\begin{framed}\vspace{-14pt}%
\begin{verbatim}
  float*** a;
  a = new float**[dim0];
  for (int i=0;i<dim0;i++) {
    a[i] = new float*[dim1];
    for (int j=0;j<dim1;j++) 
      a[i][j] = new float[dim2];
  }
\end{verbatim}%
\vspace{-12pt}\end{framed}\vspace{-5pt}%
Drawbacks of this solution are the non-contiguous buffer for the
elements (so it's unusable for many libraries) and having to keep
track of array dimensions.
At first, there seems to be no shortage of libraries to fill this
lack of \cxx\ support for dynamic multi-dimensional arrays, such as
\begin{itemize}\itemsep 0pt \parskip 0pt
\item Blitz++;
\item The Boost Multidimensional Array Library (\texttt{boost::multiarray}); 
\item Eigen;
\item Armadillo; and
\item Nested \texttt{vector}s from the Standard Template Library.
\end{itemize}
These typically do have some runtime overhead compared to the above
textbook solution, or do not allow arbitrary ranks. In contrast, the purpose of the rarray
library is to be a minimal interface for runtime multidimensional arrays with
\emph{minimal to no performance overhead} compared to the textbook solution.

\noindent{\bf Example:\vspace{-7pt}}
\begin{framed}\vspace{-14pt}%
%TEST THIS
\begin{verbatim}
  #include "rarray.h"
  int main() 
  {
     rarray<float,3> a(256, 256, 256);
     a[1][2][3] = 105;
  }
\end{verbatim}%
%END TEST THIS
\vspace{-14pt}%
\end{framed}

\noindent{\bf Design Points}

\begin{enumerate}\itemsep1pt\parskip3pt
 
\item To have dynamically allocated multidimensional arrays that
combine the convenience of automatic c++ arrays with that of the
typical textbook dynamically allocated pointer-to-pointer
structure. 

The compatibility requirement with pointer-to-pointer structures
is achieved by allocating a pointer-to-pointer structure. This
accounts for most of the memory overhead of using rarray.

\item To be as fast as pointer-to-pointer structures.

\item To have rarrays know their sizes, so that can be passed to
functions as a single argument. 

\item To enable interplay with libraries such as BLAS and LAPACK: this
  is achieved by guarranteeing contiguous elements in the
  multi-dimensional array, and a way to get this data out.

Relatedly, it should be allowed to use an existing buffer.
    
The guarrantee of contiguity means strided arrays are not supported.

\item To optionally allow bounds checking.

\item To avoid cluttered sematics around const when converting to  pointer-to-pointer structers.
\end{enumerate}

\noindent{\bf Features of rarray:\vspace{-3pt}}
\begin{itemize}\itemsep0pt\parskip2pt
\item Can use any data type {\tt T} and any rank {\tt R}.
\item Views on existing contiguous arrays.
\item Does shallow copies by default, but also has a deep {\tt copy} method.
\item Elements are accessible using repeated square brackets, like C/C++ arrays.
\item Can be emptied with the {\tt clear} method.
\item Can be filled with a uniform value with the {\tt fill} method.
\item Can be reshaped.
\item Automatic C-style arrays can be converted to rarrays using {\tt RARRAY}.
\item Checks index bounds if the preprocessor
  constant {\tt RA\_BOUNDSCHECK} is defined. 
\item A method {\tt is\_clear} to check if the rarray is uninitialized.
\item A method to get the number of elements in each
  dimension (\texttt{extent}), or in all dimensions (\texttt{shape}).
\item A method to obtain the total number of elements in the
  array (\texttt{size}).
\item A method to make the data type of the array const
  (\texttt{const\_ref}).
\item Conversion methods using the member
  function \texttt{data()} for conversions to a \texttt{T*} or
  \texttt{const T*}, using the method \texttt{ptr\_array()} for
  conversions to \texttt{T*const*} or \texttt{const T*const*}, and
  using the method \texttt{noconst\_ptr\_array()} for the conversion to a
  \texttt{T**}.
\item Streaming input and output.
\end{itemize}

\section{Using the {\tt rarray}s}

\subsection{Defining a multidimensional rarray}

To use \texttt{rarray}, first include the header file rarray.h:
\begin{framed}\vspace{-14pt}%
\begin{verbatim}
  #include "rarray.h"
\end{verbatim}%
\vspace{-14pt}
\end{framed}
This defines the (template) classes \texttt{rarray{\tt<}T,R{\tt>}}, where
\texttt T is the element type, and \texttt R is the
rank (a positive integer).  Instances can now be
declared as follows:
\begin{framed}\vspace{-18pt}%
\begin{verbatim}
  rarray<float,3> s(256,256,256);
  s[1][2][3] = 105;
  // do whatever you need with s
\end{verbatim}%
\vspace{-14pt}
\end{framed}
\noindent
or, using an external, already allocated buffer, as
\begin{framed}\vspace{-18pt}%
\begin{verbatim}
  float* data=new float[256*256*256];  
  rarray<float,3> s(data,256,256,256);
  s[1][2][3] = 105;
  // do whatever you need with s
  delete[] data;
\end{verbatim}%
\vspace{-14pt}
\end{framed}
Without the \texttt{delete[]} statement in the latter example, there would be a memory leak. This reflects that rarray is in this case not responsible
for the content. The data pointer can also be retrieved using
\texttt{s.data()}. 

Even in the former case (\texttt{rarray<float,3> s(256,256,256)}), one can release the memory of this array by calling \texttt{s.clear()} before the rarray goes out of scope.

The construction specifying a number of extents as arguments works for arrays with rank up to and including 11. For arrays with larger rank, you have to pass an pointer to the array of extents.

\subsection{Accessing the elements}

The elements of rarray objects are accessed using the repeated square
bracket notation as for automatic \cxx\ arrays (by design). Thus, if \texttt{s} is a \texttt{rarray} of rank \texttt R, the elements are accessed using \texttt{R} times an index of the form \texttt{[n$_i$]}, i.e. \texttt{s[n$_0$][n$_1$]\dots[n$_{\texttt{R}-1}$]}
For example:\vspace{-9pt}
\begin{framed}\vspace{-18pt}%
\begin{verbatim}
  for (int i=0;i<s.extent(0);i++)
    for (int j=0;j<s.extent(1);j++)
      for (int k=0;k<s.extent(2);k++)
        s[i][j][k] = i+j+k;
\end{verbatim}%
\vspace{-12pt}
\end{framed}%\vspace{-8pt}\noindent

\subsection{Streaming input and output}

Although it is usually preferrable to store large arrays in binary, rarray does provide streaming operators \texttt{<<} and \texttt{>>} to read and write to standard C++ iostreams.  The output produced by the output streaming operator is like that of automatic arrays initializers: Each dimension is started and ended by \texttt{\{} and \texttt{\}} and components are comma separated. 
E.g.:\vspace{-9pt}
\begin{framed}\vspace{-18pt}%
\begin{verbatim}
  int buf[6] = {1,2,3,4,5,6};
  rarray<int,2> arr(buf,3,2)
  std::cout << arr;
\end{verbatim}%
\vspace{-12pt}
\end{framed}\vspace{-8pt}\noindent
will print \texttt{\{\{1,2\},\{3,4\},\{5,6\}\}}. The streaming operators use those of the element types. 

Reading would not be unambiguous for some types, e.g. for \texttt{std::string}, that can contain the syntactic elements `\texttt{\{}', `\texttt{\}}', `\texttt{,}', or `\texttt{\#}'.  If thse elements are found in the output of an element, the element is prepended with a string that encodes the string length of the output. This prepending string starts with a '\texttt\#' character, then the length of the string output for the element  (excluding the prepended part), followed by a `\texttt:' character.

The input stream operators expect the same format as input.

\subsection{Copying and function arguments}

In C++, when we copy a variable of a built-in type to a new variable, the new copy is completely independent of the old variable. Likewise, the default way of passing arguments to a function involves a complete copy for built-in types.  For C-style arrays, however, only the pointer to the first element gets copied, not the whole array. The latter is called a shallow copy. Rarrays use shallow copies much like pointers, with the additional functionality that memory allocated by the rarray gets released. 

What does this essentially mean? Well:
\begin{enumerate}
\item You can pass rarrays by value to function, which is as if you were passing a pointer.
\item When you assign one rarray to another, the other simply points to the old one.
\item If you wish to do a deep copy, i.e., create a new array independent of the old array, you need to use the \texttt{copy} method.
\end{enumerate}

\subsection{Reshaping the rarray}

To use the data in an rarray but access it in a different `shape', one can
\begin{enumerate}
  \item create a new rarray which uses the data from the first rarray. E.g.
\vspace{-9pt}
\begin{framed}\vspace{-3pt}%
\begin{verbatim}
void dump(const rarray<double,3>& r) {
   rarray<double,1> rflat(r.data(), r.size());
   for (int i=0;i<r.size();i++)
       std::cout << rflat[i] << ' ';
  std::cout << std::endl;
}
\end{verbatim}%
\vspace{-12pt}
\end{framed}\vspace{-8pt}
\item or one can reshape the existing rarray with the desired dimensions. E.g.
\vspace{-9pt}
\begin{framed}\vspace{-3pt}%
\begin{verbatim}
void dump(rarray<double,3> r) {
   r.reshape(r.data(), 1, 1, r.size());
   for (int i=0;i<r.size();i++)
       std::cout << rflat[0][0][i] << ' ';
  std::cout << std::endl;
}
\end{verbatim}%
\vspace{-12pt}
\end{framed}\vspace{-8pt}
This only works if the new and old shape have the same ranks.
\end{enumerate}
The last example has no side effects, i.e., the original rarray that was passed to the dump function retains its shape. 

There is an important restriction to using reshape, namely, that only
shallow copies of rarrays can be reshaped with the \texttt{reshape}
method.  Original rarrays cannot. Because in the above example,
\texttt{r} is an argument, it is a shallow copy of the array, and can
be reshaped. If the argument were of reference type
(e.g. \texttt{rarray<double,3>\& r}), then the reshape would not be
allowed.  The reason for this is that reshaping an original array
would make any exisiting shallow copies invalid. 

Note that in both methods, the total size of the new shape should be
less or equal to that of the old shape, because no additional data is
allocated by the reshape method.

\subsection{Optional bounds checking}

If the preprocessor constant \texttt{{\tt RA\_BOUNDSCHECK}} is defined, an
\texttt{out\_of\_bounds} exception is thrown~if
\begin{itemize}\itemsep0pt\parskip3pt
\item an index is too small or too large;
\item the size of dimension is requested that does not exist (in a call to \texttt{extent(int i)});
\item a constructor is called with a zero pointer for the buffer or for the dimensions array;
\item a constructor is called with too few or too many arguments (for \texttt{R}$<=11$).
\end{itemize}
\texttt{{\tt RA\_BOUNDSCHECK}} can be defined by
adding the {\tt -DRA\_BOUNDSCHECK} argument to the compilation command, or
by \texttt{{\tt\#define RA\_BOUNDSCHECK}} before
the \texttt{{\tt\#include "rarray.h"}} in the source.
%\\
%Regardless of \texttt{{\tt RA\_BOUNDSCHECK}}, an exception is raised if the %memory allocation for rarray fails.

\section{Comparison with standard alternatives}

The more-or-less equivalent automatic array version 
\vspace{-5pt}\begin{framed}\vspace{-14pt}%
\begin{verbatim}
  float arr[256][256][256]; 
  arr[1][2][3] = 105;
\end{verbatim}
\vspace{-14pt}\end{framed}
\noindent
is a little simpler, but automatic arrays are allocated on the stack,
which is typically of limited size. Another drawback is that this array cannot be passed to functions that do
not hard-code exactly matching dimensions except for the last
one.

Using rarray also has benefits over another common \cxx\ 
solution using the STL:
\vspace{-5pt}\begin{framed}\vspace{-14pt}%
%TEST THIS
\begin{verbatim}
  #include <vector>
  int main() {
     using std::vector;
     int n = 256;                // size per dimension
     vector<vector<vector<float> > > v(n);// allocate for top dimension
     for (int i=0;i<n;i++) {
        v[i].reserve(n);         // allocate vectors for middle dimension
        for (int j=0;j<n;j++) 
           v[i][j].reserve(n);   // allocate elements in last dimension
     }
     v[1][2][3] = 105;           // assign to element (for example)
  }
\end{verbatim}%
%END TEST THIS
\vspace{-14pt}\end{framed}\vspace{-8pt}
\noindent
which is complicated, is non-contiguous in memory, and likely
slower. 

%\pagebreak[4]
\section{Class definition}

\subsection{Interface}
Effectively, the interface part of a rarray object is defined as follows:%
\vspace{-8pt}%
\begin{framed}\vspace{-14pt}%
\begin{verbatim}
template<typename T, int R>
class rarray {
public:    
 rarray();                               // create uninitialized array
 rarray(int extent0, ...);               // constructor for R<=11
 rarray(const int* extents);             // alternative: extents in array
 rarray(T* data, int extent0, ...);      // construct with existing data
 rarray(T* data, const int* extents);    // alternative: extents in array
 rarray(const rarray<T,R> &a);           // shallow copy constructor   
 ~rarray();                              // destructor 
 void         reshape(int extent0, ...); // change shape (not data)
 void         reshape(const int*extent); // change shape (not data)
 void         clear();                   // release memory
 void         fill(const T& value);      // fill with uniform value
 bool         is_clear();         const; // check if uninitialized;
 rarray<T,R>  copy()              const; // deep copy
 const int*   shape()             const; // all extents as C-style array
 int          extent(int i)       const; // extent in dimension i
 int          size()              const; // total number of elements
 T*           data()              const; // start of internal buffer
 T*const*...  ptr_array();        const; // convert to a T*const*... 
 T**...       noconst_ptr_array() const; // converts to a T**... 
 rarray<const T,R>&  const_ref()  const; // convert to const elements
 rarray<T,R>& operator=(const rarray<T,R> &a);// shallow assignment
 rarray<const T,R-1> operator[](int i) const; // element access
 rarray<T,R-1>       operator[](int i);       // element access
};
\end{verbatim}
\end{framed}



\subsection{Methods}

%\subsubsection{Get size in each dimension: \tt extent}
\noindent\textbf{Definition:} \texttt{int extent(int i) const;}

This method returns the size of dimension \texttt{i}.

%\subsubsection{Get size in each dimension: \tt extent}
\noindent\textbf{Definition:} \texttt{const int* shape() const;}

This method returns the size of all dimensions as a c array. 

%\subsubsection{Get total size: \tt size}
\noindent\textbf{Definition:} \texttt{int size() const;}

This method returns the total number of elements.

%\subsubsection{Get pointer equivalent: \tt data}

\noindent\textbf{Definition:} \texttt{T* data() const;}

This method returns a pointer to a \texttt{T*}. Const-correct.

%\subsubsection{Get pointer equivalent: \tt ptr\_array}

\noindent\textbf{Definition:} \texttt{T*const*... ptr\_array() const;}

This method returns a pointer to a \texttt{T*const*...}. Const-correct.

%\subsubsection{Get const-casted pointer equivalent: \tt noconst_ptr\_array}

\noindent\textbf{Definition:} \texttt{T**... noconst\_ptr\_array() const;}

This method returns a pointer to a \texttt{T**...}. This conversion breaks const-correctness.%, as the pointer structure itself could in principle be modified.

%\subsubsection{Get const-content equivalent: \tt const\_ref}

\noindent\textbf{Definition:} \texttt{rarray{\tt<}const T,R{\tt>}\& const\_ref() const;}

This method creates reference to this with const elements.

\noindent\textbf{Definition:} \texttt{void reshape(int extent0, ...);}

This method can be used to change the shape of the rarray. The data in the underlying buffer will not change. Note that the number of dimensions must remain the same and the total new size must be less or equal to the old size.  This works only for arrays with rank up to and including 11.  Warning: if the rarray was a shallow copy of another rarray, that original array will be resized as well! 

\noindent\textbf{Definition:} \texttt{void reshape(const int* extents);}

Same as the previous reshape method, but takes an array of new dimensions as an argument.  This is the only way to reshape an array of rank seven or higher.

\noindent\textbf{Definition:} \texttt{void clear();}

This method release memory the memory associated with the rarray. If the rarray was created by providing a pre-existing buffer for the data, the memory of this buffer does not released.

\noindent\textbf{Definition:} \texttt{void fill(const T\& value);}

This method fills the array with a uniform value \texttt{value}.

\noindent\textbf{Definition:} \texttt{bool is\_clear();}

Checks if the rarray is in an uninitialized state.

\noindent\textbf{Definition:} \texttt{rarray{\tt<}T,R{\tt>}\& copy() const;}

This method creates deep copy of the rarray, i.e., an independent rarray with the same dimensions are the original rarray and with its content a copy of the original rarray's content.

\section{Returning a rarray from a function}

Unless you're using rarray in a C++11 context, the shallow copying of
rarrays causes problems when returning a rarray from a function.

Consider the function \texttt{zeros} used in \texttt{main()}:
\vspace{-9pt}
\begin{framed}\vspace{-12pt}%
\begin{verbatim}
rarray<double,2> zeros(int n, int m) {
   rarray<double,2> r(n,m);
   r.fill(0.0);
   return r;
}
int main() {
   rarray<float,2> s = zeros(100,100);
   return s[99][99];
}
\end{verbatim}%
\vspace{-12pt}
\end{framed}\vspace{-8pt}
In line 2, a rarray 'r' is created, and filled, on line 3, with zeros.
On line 4, this array is shallowly copied out of the function and into
's' on line 7. Once copied into 's', 'r' is destroyed and releases its
memory, but this leaves 's' in an invalid state (as it points to
memory that is no longer allocated.

Instead, we would like the rarray 's' to become responsible for the
memory deallocation, i.e., we'd like r to become a shallow copy and s
the original.  To do so, rarray.h provides the \texttt{RA\_MOVE}
macro, which affectively transfers the ownership. In the above
example, it should be used as follows:
\vspace{-9pt}
\begin{framed}\vspace{-12pt}%
\begin{verbatim}
rarray<double,2> zeros(int n, int m) {
   rarray<double,2> r(n,m);
   r.fill(0.0);
   return RA_MOVE(r);
}
\end{verbatim}%
\vspace{-12pt}
\end{framed}\vspace{-8pt}

In C++11, this kind of behavior can (and will) be implemented using
\emph{move semantics}, which make the first form of the \texttt{zeros}
function work as correctly.

\section{Conversions}

\subsection{Converting automatic C-style arrays to rarrays}

It is possible to convert C-style automatic arrays to rarrays using the {\tt RARRAY} macro (which calls some templated functions under the hood).  You can apply {\tt RARRAY} to any automatic array of rank six or less, and even to a rarray (which essentially does nothing). 
The main convenience of this is that one can write functions that take rarray argument(s) and pass automatic arrays to them. Another use is in initializing a rarray. For example:
\vspace{-5pt}\begin{framed}\vspace{-14pt}%
%TEST THIS
\begin{verbatim}
#include <iostream>
#include "rarray.h"
void print2d(const rarray<float,2> &s) {
    for (int i=0; i<s.extent(0); i++) {
        for (int j=0; j<s.extent(1); j++)
            std::cout << s[i][j] << ' ';
        std::cout << std::endl;
    }
}
int main() {
    float printme[4][4] = { { 1.0, 1.2, 1.4, 1.6},
                            { 2.0, 2.2, 2.4, 2.6},
                            { 3.0, 3.2, 3.4, 3.6},
                            { 4.0, 4.2, 4.4, 4.6} };
    print2d(RARRAY(printme));
    rarray<float,2> a = RARRAY(printme).copy();
    print2d(a);
}
\end{verbatim}%
%END TEST THIS
\vspace{-14pt}\end{framed}\vspace{-8pt}



\subsection{Conversions for function arguments}

A function might take a \texttt{rarray{\tt<}const T,R{\tt>}} parameter if elements are not changed by it. Because \cxx\ cannot convert a reference to a \texttt{rarray{\tt<}T,R{\tt>}} to a \texttt{rarray{\tt<}const T,R{\tt>}}, one has to use the \texttt{const\_ref} method to do this for you.
For example:
\vspace{-5pt}\begin{framed}\vspace{-14pt}%
%TEST THIS
\begin{verbatim}
  float add(const rarray<const float,2> &s)   {
     float x = 0.0;
     for (int i=0; i<s.extent(0); i++)
        for (int j=0; j<s.extent(1); j++)
           x += s[i][j];
     return x;
  }
  int main() {
     rarray<float,2> s(40, 40);
     float z = add(s.const_ref()); // because add() takes <const float>
  }
\end{verbatim}%
%END TEST THIS
\vspace{-14pt}
\end{framed}\vspace{-8pt}

rarray objects are also easy to pass to function that do not use \texttt{rarray}s. Because there are, by design, no automatic conversions of a
rarray, this is done using methods.

There are two main ways that such functions expect a multidimensional
array to be passed: either as a pointer (a \texttt{T*}) to the first
element of the internal buffer composed of all elements, or as a
pointer-to-pointer structure (a \texttt{T**...}).  In the former case, it may be
important to know that a rarray stores elements in row-major
format.

With the \texttt{const} keyword, the number of useful \cxx\ forms for multidimensional array arguments has grown to about six.  In the case of a two-dimensional array these take the forms:
 \texttt{T*}, \texttt{const T*}, \texttt{T*const*}, \texttt{const T*const*}, \texttt{T**}, and \texttt{const T**}.
Using the rarray library, const-correct argument passing requires the \texttt{data} or \texttt{ptr\_array} method but non-const-correct argument passing will require the \texttt{noconst\_ptr\_array} function, possibly combined with \texttt{const\_ref}.
We will briefly looking at these cases separately now.
%  Each will be started with a simple example, followed by an explanation.

\subsection{Conversion to a {\tt T*} or a {\tt const T*}}

A function may expect a multidimensional array to be passed as a
simple pointer to the first element, of the form \texttt{T*}, or of the form  \texttt{const T*}. A rarray object
\texttt{s} of type \texttt{rarray{\tt<}T,I{\tt>}} can be passed to these
functions using the syntax \texttt{s.data()}, which yields a
\texttt{T*}.

\noindent{\bf Example 1:}
\begin{framed}\vspace{-15pt}%
%TEST THIS
\begin{verbatim}
  void fill1(float* a, int n1, int n2, float z) {
     for (int i=0; i<n1*n2; i++)
       a[i] = z;
  }
  int main() {
     rarray<float,2> s(40, 40);
     fill1(s.data(), s.extent(0), s.extent(1), 3.14);
  }
\end{verbatim}%
%END TEST THIS
\end{framed}

\pagebreak[3]

\noindent{\bf Example 2:}
\vspace{-5pt}\begin{framed}\vspace{-14pt}%
%TEST THIS
\begin{verbatim}
  float add2(const float* a, int n1, int n2) {
     float x = 0.0;
     for (int i=0; i<n1*n2; i++)
         x += a[i];
     return x;
  }
  int main() {
     rarray<float,2> s(40, 40);
     float z = add2(s.data(), s.extent(0), s.extent(1));
  }
\end{verbatim}%
%END TEST THIS
\vspace{-14pt}\end{framed}\vspace{-8pt}

\noindent
C++ accepts a \texttt{float*} instead of a \texttt{const float*}, so \texttt{data()} can be used in the latter example.

\subsection{Conversion to a {\tt T*const*} or a {\tt const T*const*} }

\noindent
In \texttt{T*const*}, the middle const means that one cannot reassign the row pointers.
The rarray classes can be converted to this type using the \texttt{ptr\_array()} method. 
For higher dimensions, this case generalizes to \texttt{T*const*const*}, \texttt{T*const*const*const*}, etc.

\noindent
{\bf Example 1:}
\vspace{-5pt}\begin{framed}\vspace{-14pt}%
%TEST THIS
\begin{verbatim}
  void fill3(float*const* a, int n1, int n2, float z) {
     for (int i=0; i<n1; i++)
        for (int j=0; j<n2; j++)
           a[i][j] = z;
  }
  int main() {
     rarray<float,2> s(40, 40);
     fill3(s.ptr_array(), s.extent(0), s.extent(1), 3.14);
  }
\end{verbatim}%
%END TEST THIS
\vspace{-14pt}\end{framed}\vspace{-8pt}

\noindent
{\bf Example 2:}
\vspace{-5pt}\begin{framed}\vspace{-14pt}%
%TEST THIS
\begin{verbatim}
  float add4(const float*const* a, int n1, int n2) {
     float x = 0.0;
     for (int i=0; i<n1; i++)
        for (int j=0; j<n2; j++)
           x += a[i][j];
     return x;
  }
  int main() {
     rarray<float,2> s(40, 40);
     float z = add4(s.ptr_array(), 40, 40);
  }
\end{verbatim}%
%END TEST THIS
\vspace{-14pt}
\end{framed}

\noindent
\cxx\ accepts a \texttt{T*const*} where a \texttt{const T*const*} is expected, so here one can again use the method \texttt{ptr\_array()}.
This extends to its generalizations \texttt{const T*const*const*}, \texttt{const T*const*const*const*}, etc., as well.


\subsection{Conversion to a {\tt T**}}

\noindent
Generating a T** from a rarray object, one could change the internal structure of that rarray object through the double pointer.  This is therefore considered a not const-correct.  It is however commonly needed, so \texttt{rarray} does have a function for it, called \texttt{noconst\_ptr\_array}. 

\noindent
{\bf Example:}
\vspace{-5pt}\begin{framed}\vspace{-14pt}%
%TEST THIS
\begin{verbatim}
  void fill5(float** a, int n1, int n2, float z) {
     for (int i=0; i<n1; i++)
        for (int j=0; j<n2; j++)
           a[i][j] = z;
  }
  int main() {
     rarray<float,2> s(40, 40);
     fill5(s.noconst_ptr_array(), s.extent(0), s.extent(1), 3.14);
  }
\end{verbatim}%
%END TEST THIS
\vspace{-14pt}
\end{framed}

\subsection{Conversion to a {\tt const T**}}
\noindent
C++ does not allow conversion from T** to \texttt{const T**}. To convert to a \texttt{const T**}, one first needs to convert the \texttt{rarray{\tt<}T,R{\tt>}} to a \texttt{rarray{\tt<}const T,R{\tt>}} using \texttt{const\_ref()}, after which one can use the \texttt{noconst\_ptr\_array} function.

\noindent
{\bf Example:}
\vspace{-5pt}\begin{framed}\vspace{-14pt}%
%TEST THIS
\begin{verbatim}
  float add6(const float** a, int n1, int n2) {
     float x = 0.0;
     for (int i=0; i<n1; i++)
        for (int j=0; j<n2; j++)
           x += a[i][j];
     return x;
  }
  int main() {
     rarray<float,2> s(40, 40);
     float z = add6(s.const_ref().noconst_ptr_array(), 40, 40);
  }
\end{verbatim}%
%END TEST THIS
\vspace{-14pt}
\end{framed}

\section{Overheads}

\subsection{Memory overhead using the rarray class}

The memory overhead here comes from having to store the dimensions and a pointer-to-pointer structure.  The latter account for most of the memory overhead.   A rarray object of 100$\times$100$\times$100$\times$100  doubles on a 64-bit machine will have a memory overhead of a bit over 1\%. In general, the memory overhead as a percentage is roughly 100\% divided by the last dimension. Therefore, avoid rarrays with a small last dimension such as 100$\times$100$\times$100$\times$2.

\subsection{Performance overhead using the rarray class}

When compiled with optimization on (\texttt{-O3 -fstrict-aliasing} is a good default), inlining should make the rarray objects at least as fast as the textbook solution above, for most compilers (tested with gcc 4.6.1 -- 4.8.2 and, icc 12.1.5 -- 13.1.1 on x86, and with xlC 12.1 on power 6).
So in that sense, no there is no performance overhead.  However, with \texttt{{\tt RA\_BOUNDSCHECK}} defined, one incurs a compiler and machine dependent performance hit.

Let us give a bit more details as to why there should not be any performance overhead. Temporary intermediate expression in a multi-bracketed expression (such as a[5] in a[5][6][7]) are represented by an intermediate class called rarray\_intermediate. This prevents assignment to such expressions and allows the optional bound checking to work.  Most compilers (gcc, intel) can optimize these intermediate classes away when bounds checking is off. However
some compilers (e.g. pgi) cannot do this and suffer from speed degredation because of these intermediate objects.

To get full speedup for these less-optimizing compilers, you can define the preprocessor constant \texttt{RA\_SKIPINTERMEDIATE} (usually with a \texttt{-DRA\_SKIPINTERMEDIATE} compiler option).  Intermediate expression in a multi-bracketed expression are then replaced by bare pointers-to-pointers.

Note that when \texttt{RA\_BOUNDSCHECK} is set, it will switch off \texttt{RA\_SKIPINTERMEDIATE}.

\subsection{Compilation overhead using the rarray class}

There is an overhead in the compilation stage, but this is very compiler dependent.


\end{document}
