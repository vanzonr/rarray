\documentclass[11pt,twoside]{article}
\usepackage{geometry,fancyhdr,hyperref,framed,charter,fullpage}
\setcounter{tocdepth}{1}
\pagestyle{fancy}
\renewcommand{\sectionmark}[1]{\markright{\thesection\quad #1}}
\fancyhf{}
\fancyhead[LE,RO]{\bfseries\thepage}
\fancyhead[LO]{\bfseries \hfill\rightmark\hfill}
\fancyhead[RE]{\hfill\bfseries\rightmark \hfill}
\renewcommand{\headrulewidth}{0.5pt}
\renewcommand{\footrulewidth}{0pt}
\addtolength{\headheight}{15pt}
\fancypagestyle{plain}{\fancyhead{}\renewcommand{\headrulewidth}{0pt}}
\let\oldtt=\tt
%\renewcommand{\texttt}[1]{\color{blue}\tt #1\color{black}}
\newcommand{\cxx}{C{++}}

%\renewcommand{\texttt}[1]{#1}
\begin{document}

\setlength{\parskip}{1mm}

\title{\texttt{rarray}: Design of a Reference-Counted Multidimensional Arrays for \cxx}

\author{Ramses van Zon} 

\date{January 2024}

\maketitle

\section{Introduction}

While C and thus \cxx\ has some support for multidimensional arrays
whose sizes are known at compile time, the support for arrays with
sizes that are known only at runtime, is limited. For one-dimensional
arrays,  \cxx\ has a reasonable allocation and deallocation constructs in the operators
\texttt{new} and \texttt{delete} in the standard.  A standard way to allocate a
one-dimensional array is as follows:
\vspace{-5pt}\begin{framed}\vspace{-14pt}%
\begin{verbatim}
int n = 1000;
float* a = new float[n];
a[40] = 2.4; // as an example on how to use this, access element at offset 40
delete[] a;
\end{verbatim}%
\vspace{-12pt}\end{framed}\vspace{-5pt}\noindent%
It is important to note that this code also works if \texttt{n} was
not known yet at compile time, e.g., if it was passed instead as a function
argument or read in as input.

This style of allocation with a
``raw'' pointer is discouraged in \cxx\ in favor of using ``smart''
pointers, which is possible since the C++17 standard:
\vspace{-5pt}\begin{framed}\vspace{-14pt}%
\begin{verbatim}
int n = 1000;
std::unique_ptr<float[]> a(new float[n]); 
a[40] = 2.4; // as an example on how to use this, access element at offset 40
// a gets deallocated automatically, or one can explicitly call a.reset(nullptr)
\end{verbatim}%
\vspace{-12pt}\end{framed}\vspace{-5pt}\noindent%
Automatic deallocation happens when the variable \texttt{a} goes out of scope.
A unique pointer cannot be copied.  Instead of
\texttt{unique\_ptr} one can use \texttt{shared\_ptr}, which can be
copied and keeps a reference counter to know when to deallocate the
memory.  Then, deallocation happens when the life time of all copies
of the \texttt{shared\_ptr} has ended.

In the above code snippets, the \texttt{new} construct and the
\texttt{std::unique\_ptr/std::shared\_ptr} assign the address of the
array to a pointer. These pointers do not remember their size, so they
are not really an 'array'.  The standard \cxx\ library does provide a
dynamically allocated one-dimensional array that remembers its size, in the form of the \texttt{std::vector}, e.g.
\vspace{-5pt}\begin{framed}\vspace{-14pt}%
\begin{verbatim}
int n = 1000;
std::vector a(n);
a[40] = 2.4; // as an example on how to use this, access element at offset 40
// a gets automatically deallocated, or one can explicitly call a.clear()
\end{verbatim}%
\vspace{-12pt}\end{framed}\vspace{-5pt}%

Multi-dimensional runtime-allocated arrays are currently not supported yet by
\cxx, but there is a proposal for a non-owning multidimensional array in
the C++23 standard, and C++26 may have an owning multidimenstional
array.

To handle these kinds of arrays in \cxx, the (early) textbook solution for multidimensional arrays that are
dynamically allocated during runtime, would be as follows (here for a
three-dimensional array of \texttt{dim0}$\times$\texttt{dim1}$\times$\texttt{dim2})
\vspace{-5pt}\begin{framed}\vspace{-14pt}%
\begin{verbatim}
  float*** A;
  A = new float**[dim0];
  for (int i=0;i<dim0;i++) {
    A[i] = new float*[dim1];
    for (int j=0;j<dim1;j++) 
      A[i][j] = new float[dim2];
  }
  A[1][2][3] = 105; // as an example
\end{verbatim}%
\vspace{-12pt}\end{framed}\vspace{-5pt}\noindent%
Apart from the fact this will soon be obsolute, drawbacks of this solution are:
\begin{itemize}
  \item the elements are not stored contiguously in memory, making
    this multi-dimensional array unusable
    for many numerical libraries,
  \item one has to keep track of array dimensions, and pass them along
    to functions,
  \item the intermediate pointers are non-const, so the
    internal pointer structure can be changed
    whereas, conceptually, \texttt{a} ought to be of type \texttt{float*const*const*}.
\end{itemize}
At first, there seems to be no shortage of libraries to fill this
lack of \cxx\ support for dynamic multi-dimensional arrays, such as
\begin{itemize}\itemsep 0pt \parskip 0pt
\item Blitz++;
\item The Boost Multidimensional Array Library (\texttt{boost::multiarray}); 
\item Eigen;
\item Armadillo
\item Nested \texttt{vector}s from the Standard Template Library; and
\item Kokkos's reference implementation of the C++23 mdspan template.
\end{itemize}
These typically do have some runtime overhead compared to the above
textbook solution, or do not allow arbitrary ranks. In contrast, the purpose of the rarray
library is to be a minimal interface for runtime multidimensional
arrays of
arbitrary rank with
\emph{minimal to no performance overhead} compared to the textbook
solution.  From the above list of library solutions, only the  implementation in
Kokkos of the non-owning mdspan has virtually no overhead.

Rarray aims to be simpler; all it takes to create a three-dimensional 
\noindent{\bf Example:\vspace{-7pt}}
\begin{framed}\vspace{-14pt}%
\begin{verbatim}
  #include <rarray>
  ...
     rtensor<float> A(dim0,dim1,dim2);
     A[1][2][3] = 105; // as an example
  ...
\end{verbatim}%
\vspace{-14pt}%
\end{framed}

\noindent{\bf Design Points of rarray}

\begin{enumerate}\itemsep1pt\parskip3pt
 
\item To have dynamically allocated multidimensional arrays that
combine the ease and convenience of automatic \cxx\ arrays while being compatible
with the
typical textbook-style dynamically allocated pointer-to-pointer
structure. 

The compatibility requirement with pointer-to-pointer structures
is achieved by allocating a pointer-to-pointer structure. This
structure accounts for most of the memory overhead from using \texttt{rarray}.

\item To be as fast as pointer-to-pointer structures.

\item To do shallow copy by default, deep copy possible.
  
\item To have rarrays know their sizes, so that they can be passed to
functions as a single argument. 

\item To enable interfacing with libraries such as BLAS, LAPACK and FFTW: this
  is achieved by guarranteeing contiguous elements in the
  multi-dimensional array, and a way to get this data out.

  The guarrantee of contiguity means strided arrays are not supported.

\item To allow sharing between components of an application while
  avoiding dangling references. This is possible by utilizing reference counting.

\item To allow rarrays to hold non-owning views that use an existing buffer,
  without having to use a separate type.

\item To avoid some of the cluttered sematics around \texttt{const}
  correctness when converting to pointer-to-pointer structures when
  interfacing with legacy code.

\end{enumerate}

\noindent{\bf Features of rarray:\vspace{-3pt}}
\begin{itemize}\itemsep0pt\parskip2pt
\item Can use any data type {\tt T} and any rank {\tt R}.
\item Elements are accessible using repeated square brackets, like C/C++ arrays.
\item Views on pre-allocated contiguous arrays.
\item Does shallow, (atomic) reference counted, copies by default, but also has a deep {\tt copy}
  method.
\item Use of \cxx11 move semantics for efficiency.
\item Can be emptied with the {\tt clear} method.
\item Can be filled with a uniform value with the {\tt fill} method.
\item Can be filled with a nested initializer list.
\item Can be formed from a nested initializer list.
\item Can be reshaped.
\item Automatic C-style arrays can be converted to rarrays.
\item A method {\tt empty} to check if the rarray is uninitialized.
\item A method to get the number of elements in each
  dimension (\texttt{extent}), or in all dimensions (\texttt{shape}).
\item A method to obtain the total number of elements in the
  array (\texttt{size}).
\item A method to make the data type of the array const
  (\texttt{const\_ref}).  Used in automatic conversion from
  \texttt{rarray<T,R>} to \texttt{rarray<const T,R>}.
\item Conversion methods using the member
  function \texttt{data()} for conversions to a \texttt{T*} or
  \texttt{const T*}, using the method \texttt{ptr\_array()} for
  conversions to \texttt{T*const*} or \texttt{const T*const*}, and
  using the method \texttt{noconst\_ptr\_array()} for the conversion to a
  \texttt{T**}.
\item Streaming input and output.
\item Checks index bounds if the preprocessor constant {\tt RA\_BOUNDSCHECK} is defined. 
\end{itemize}

\section{Comparison with standard alternatives}

Compared to the old textbook method of declaring an array (see above), or the rarray method:
\vspace{-5pt}\begin{framed}\vspace{-14pt}%
\begin{verbatim}
#include <rarray>
int main() {
  int n = 256;
  rarray<float,3> arr(n,n,n); 
  arr[1][2][3] = 105; // for example
}
\end{verbatim}
\vspace{-14pt}\end{framed}
\noindent
the more-or-less equivalent automatic array version 
\vspace{-5pt}\begin{framed}\vspace{-14pt}%
\begin{verbatim}
int main() {
  int n = 256;
  float arr[n][n][n]; 
  arr[1][2][3] = 105; // for example
}
\end{verbatim}
\vspace{-14pt}\end{framed}
\noindent
is slightly simpler, but automatic arrays are allocated on the stack,
which is typically of limited size. Another big drawback is that this array cannot be passed to functions that do
not hard-code exact matching dimensions except for the last
one.

Using rarray also has benefits over another \cxx\ 
solution, i.e. using the \texttt{std::vector} class from the Standard Template Library:
\vspace{-5pt}\begin{framed}\vspace{-14pt}%
\begin{verbatim}
  #include <vector>
  int main() {
     using std::vector;
     int n = 256;                // size per dimension
     vector<vector<vector<float>>> v(n);// allocate for top dimension
     for (int i=0;i<n;i++) {
        v[i].resize(n);         // allocate vectors for middle dimension
        for (int j=0;j<n;j++) 
           v[i][j].resize(n);   // allocate elements in last dimension
     }
     v[1][2][3] = 105;           // assign to element (for example)
  }
\end{verbatim}%
\vspace{-14pt}\end{framed}\vspace{-8pt}
\noindent
which is complicated, is non-contiguous in memory, and likely
slower.

C++23 will have a non-owning library, mdspan, which should work
roughly as follows:
\vspace{-5pt}\begin{framed}\vspace{-14pt}%
\begin{verbatim}
  #include <memory>
  #include <mdspan>
  int main() {
     int n = 256;                  // size per dimension
     std::unique_ptr<float[]> p (new float[n*n*n]); // or vector or a shared_ptr
     using exts = std::extents<size_t,std::dynamic_extent,
                  std::dynamic_extent,std::dynamic_extent>;
     std::mdspan<float,exts> (vector.data(), exts(n,n,n));
     v[1,2,3] = 105;               // assign to element (for example)
  }
\end{verbatim}%
\vspace{-14pt}\end{framed}\vspace{-8pt}
Which is more involved that the rarray solution and does not offer
reference counting.  

\section{Performance}

\subsection{General performance consideration}

\noindent
\texttt{rarray} is written specifically with performance in
mind. However, the way it manages to mimic the natural c and c++ way
of using multi-dimensional arrays, i.e. with (repeated) square
brackets, would come at a high cost of function calls and temporary
objects. The rarray library tries to inline most function calls to avoid this
cost, but it needs the c++ compiler's optimization capabilities to
help out. Even the lowest optimization levels, such as obtained with
the \texttt{-O1} flag, will enable this for many compilers, though some will need hight levels of optimization.

Rarray is meant to be used with higher levels of optimization. When
compiled with higher optimization levels, such as \texttt{-O3
  -fstrict-aliasing} for \texttt{g++}, inlining should make the rarray
objects at least as fast as the textbook solution for multidimensional
arrays mentioned in the introduction.

\subsection{Debugging}

When debugging a code with a symbolic debugger (e.g. gdb), in addition
to compiling with an option to include ``symbols'' (i.e., names of
functions and other information regarding the code) into
the executatble (e.g. \texttt{-g}), one usually switches off the
compiler's optimization capabilities (\texttt{-O0}), because it can
change the order of instructions in the code, which makes debugging
very difficult. As explained above, when working with rarray, turning
of the compiler's optimization options can cause the program to
become quite a bit slower, which can also hamper debugging.  If the
bug is unlikely to occur in rarray, it may be worth trying build with
a mild optimization level such as \texttt{-O1} for debugging.

If, on the other hand, the bug is in rarray, one likely needs to
switch off optimization to see what is going on.  In addition to the
\texttt{-O0} flag, this also requires switching off forced inlining,
using the \texttt{-DRA\_INLINE=inline} flag.

\subsection{Profiling}

Sampling profiling tools (such as \texttt{gprof}) work by periodically
recording the state and call stack of the program as it runs.  They
can be very useful for performance analysis. When using these tools
with rarray, it is advisable to compile with a minimum level of
optimization \texttt{-O1}. Otherwise, a lot of the internal function
calls of rarray that could simply be optimized away, and would pollute
the sampling. 

\subsection{Memory overhead using the rarray class}

The memory overhead here comes from having to store the dimensions and a pointer-to-pointer structure.  The latter account for most of the memory overhead.   A rarray object of 100$\times$100$\times$100$\times$100  doubles on a 64-bit machine will have a memory overhead of a bit over 1\%. In general, the memory overhead as a percentage is roughly 100\% divided by the last dimension. Therefore, avoid rarrays with a small last dimension such as 100$\times$100$\times$100$\times$2.

\subsection{Compilation overhead using the rarray class}

There is an overhead in the compilation stage, but this is very compiler dependent.


Finally, it is worth noting that many numerical libraries care about
the ``alignment'' of the data.  Rarray currently does not have any way of
facilitating alignment, but a user can allocate aligned
data and pass it as a pre-existing buffer to the rarray constructor.




\end{document}
